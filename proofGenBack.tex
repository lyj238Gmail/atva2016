
\subsection{The Stucture of the Generated Proof}
A formal model of a protocol in a theorem prover like Isabelle
includes the definitions of constants and rules and invariants, lemmas, and proofs
(readers can refer to~\cite{paraverifier} for detailed illustration of the formal proof script).

With the help of all the lemmas such as {\sf ruleVsinv1},
we can prove the following lemma  {\sf lemma\_inv\_1\_on\_rules}
which specifies that for all $r \in rules ~N$, and $f$ is a formula $f$
which is generated by instantiating ${\it inv1}$ with some parameters
$\iInv_1$ and $\iInv_2$, ${\sf CRS}~s~ f~ r~ (invariants~ N)$.
\JP{Make sure these are consistent with previous parts. I will check later.}

\begin{specification}
\\
lemma lemma$\_$inv1$\_$on$\_$rules:\\
assumes a1: r $\in$ rules N and \\
~~a2: ($\exists$ $\_$iInv1 $\_$iInv2. $\_$iInv1$\le$
~~N$\wedge$$\_$iInv2$\le$N$\wedge$ \\
~~\iInv1$\neq$\iInv2$\wedge$f=inv1  \iInv1 \iInv2)\\
shows CRS s f r (invariants N)\\
proof -\\
~~have ($\exists$ i. i$\le$N$\wedge$r=try  i)$\vee$ \\
~~($\exists$ i. i$\le$N$\wedge$r=crit  i)$\vee$ \\
~~($\exists$ i. i$\le$N$\wedge$r=exit  i)$\vee$\\
~~($\exists$ i. i$\le$N$\wedge$r=idle  i)\\
apply (cut\_tac a1, auto) done\\
moreover \{assume b1: ($\exists$ i. i$\le$N$\wedge$r=try  i)\\
~~have invHoldForRule' s f r (invariants N)\\
~~apply (cut$\_$tac a2 b1, metis tryVsinv1) done\}\\
moreover \{assume a1: ($\exists$ i. i$\le$ N$\wedge$r=crit  i)\\
~~have invHoldForRule' s f r (invariants N)\\
~~apply (cut\_tac a2 b1, metis critVsinv1) done\}\\
moreover \{assume a1: ($\exists$ i. i$\le$ N$\wedge$r=exit  i)\\
~~have invHoldForRule' s f r (invariants N)\\
~~apply (cut\_tac a2 b1, metis exitVsinv1) done\}\\
moreover \{assume a1:($\exists$ i. i$\le$N$\wedge$r=idle  i)\\
~~have invHoldForRule' s f r (invariants N)\\
~~apply (cut\_tac a2 b1, metis idleVsinv1) done\}\\
ultimately show invHoldForRule' s f r \\
~~ (invariants N) by auto\\
qed\\
\end{specification}
\JP{While cleaning up the latex source codes, I feel that in the above proof
there are many typos which I don't know how to fix.}

With the help of all the lemmas such as {\sf lemma$\_$inv$_i\_$on$\_$rules},
we can prove the following lemma  {\sf invs$\_$on$\_$rules} which
specifies that for all $f \in invariants~ N$ and $r \in rules~ N$,
$invHoldForRule ~s~ f~ r~ (invariants~ N)$.
\JP{In this section, I feel that we are mixing Isabelle codes and our formalisations
as described in previous sections. We need to discuss what to do.
In the above paragraph, latex source codes should be cleaned up.}

\begin{specification}
\\
lemma invs\_on\_rules: \\
assumes a1: f $\in$ invariants N  and \\
~~a2: r $\in$ rules N\\
shows invHoldForRule' s f r (invariants N)\\
proof -\\
have b1: ($\exists$ \iInv1 \iInv2. \iInv1$\le$N$\wedge$\\
~~~~~\iInv2$\le$N$\wedge$\iInv1$\neq$\iInv2$\wedge$\\
~~f=inv1  \iInv1 \iInv2)$\vee$\\
~~($\exists$ \iInv2. \iInv2$\le$N$\wedge$f=inv2  \iInv2)$\vee$\\
~~($\exists$ \iInv1 \iInv2. \iInv1$\le$N$\wedge$\iInv2$\le$N$\wedge$\iInv1$\neq$\iInv2$\wedge$\\
~~f=inv3  \iInv1 \iInv2)$\vee$\\
~~($\exists$ \iInv2. \iInv2$\le$N$\wedge$f=inv4  \iInv2)$\vee$\\
~~($\exists$ \iInv1 \iInv2. \iInv1$\le$N$\wedge$\iInv2$\le$N$\wedge$\iInv1$\neq$\iInv2$\wedge$\\
~~f=inv5  \iInv1 \iInv2)\\
apply (cut\_tac a1, auto) done\\
moreover \{assume b1:($\exists$ \iInv1 \iInv2. \iInv1$\le$N$\wedge$ \\
~~~~\iInv2$\le$ N$\wedge$\iInv1$\neq$\iInv2$\wedge$f=inv1  \iInv1 \iInv2)\\
~~have invHoldForRule' s f r (invariants N)\\
~~apply (cut\_tac a2 b1,\\
~~~~metis lemma\_inv1\_on\_rules) done\}\\
moreover \{assume b1:($\exists$ \iInv2. \iInv2$\le$ N\\
~~~~$\wedge$f=inv2  \iInv2)\\
~~have invHoldForRule' s f r (invariants N)\\
~~apply (cut\_tac a2 b1,\\
~~~~metis lemma\_inv2\_on\_rules) done\}\\
moreover \{assume b1:($\exists$ \iInv1 \iInv2. \iInv1$\le$N$\wedge$\\
~~\iInv2$\le$ N$\wedge$\iInv1$\neq$\iInv2$\wedge$f=inv3  \iInv1 \iInv2)\\
~~have invHoldForRule' s f r (invariants N)\\
~~apply (cut\_tac a2 b1,\\
~~~~metis lemma\_inv3\_on\_rules) done\}\\
moreover \{assume b1:($\exists$ \iInv2. \iInv2$\le$N$\wedge$ \\
~~f=inv4  \iInv2)\\
~~have invHoldForRule' s f r (invariants N)\\
~~apply (cut\_tac a2 b1,\\
~~~~metis lemma\_inv4\_on\_rules) done\}\\
moreover \{assume b1:($\exists$ \iInv1 \iInv2. \iInv1$\le$N$\wedge$\\
~~\iInv2$\le$ N$\wedge$\iInv1$\neq$\iInv2$\wedge$f=inv5  \iInv1 \iInv2)\\
~~have invHoldForRule' s f r (invariants N)\\
~~apply (cut\_tac a2 b1,\\
~~~~metis lemma\_inv5\_on\_rules) done\}\\
ultimately show invHoldForRule' s f r\\
~~~~(invariants N)\\
~~apply fastforce done\\
qed\\
\end{specification}

%------------------------------------------------------
\medskip\noindent
{\bf Lemmas on initial states.}
%------------------------------------------------------
We first discuss the definition on the initial state of the protocol,
and the lemmas specifying that each invariant formula holds at the initial state.
A typical Isabelle definition on the initial state of the protocol is shown as follows:\\

\begin{specification}
\\
definition initSpec0::nat $\Rightarrow$ formula where [simp]:\\
initSpec0 N $\equiv$ (forallForm (down N) \\
(\% i. (eqn (IVar (Para (Ident "n") i)) (Const I))))\\

definition initSpec1::formula where [simp]:\\
initSpec1  $\equiv$ (eqn (IVar (Ident ''x'')) (Const true))\\

definition allInitSpecs::nat $\Rightarrow$ formula list\\
allInitSpecs N $\equiv$ [(initSpec0 N),(initSpec1 )]\\

lemma iniImply\_inv4:\\
assumes a1: ($\exists$\iInv1. \iInv1$\le$N$\wedge$f=inv4 \iInv1)\\
and a2: formEval (andList (allInitSpecs N)) s\\
shows formEval f s\\
 using a1 a2 by auto\\
\end{specification}

\JP{The above Isabelle code fragment is very difficult to read.}
${\tt initSpec0}$ and ${\tt initSpec1}$ specifies the assignments
on each variable ${\tt a[i]}$ where ${\tt i \le N}$ and ${\tt x}$.
\JP{What is ${\tt x}$?}
The specifications of the initial state is the list of all the specification definition on related state variables.
Lemma ${\tt iniImply\_inv4}$ simply specifies that the invariant formula$ {\tt inv4}$
holds at a state ${\tt s}$ which satisfies the conjunction of the specification of the initial state.
Isabelle's ${\tt auto}$ method can solve this goal automatically.
Other lemmas specifying that other invariant formulas hold at the initial state are similar.

With the lemmas such as ${\tt iniImply\_inv4}$,
for any invariant $inv \in (\mathsf{invariants} ~N) $,  any
state $s$, if $ini$ is evaluated true at state $s$, then $inv$ is
evaluated true at state $s$.\\

\begin{specification}\\
lemma on$\_$inis:
assumes  a1: f $\in$ (invariants N) and \\
~~a2: ini $\in$ \{andList (allInitSpecs N)\} and \\
~~a3: formEval ini s\\
shows formEval f s\\
proof -\\
have c1: ($\exists$ \iInv1 \iInv2. \iInv1$\le$N$\wedge$ \\
~~\iInv2$\le$N$\wedge$\iInv1$\neq$\iInv2$\wedge$\\
~~f=inv$\_$$\_$1  \iInv1 \iInv2)$\vee$\\
~~($\exists$ \iInv2. \iInv2$\le$N$\wedge$f=inv$\_$$\_$2  \iInv2)$\vee$\\
~~($\exists$ \iInv1 \iInv2. \iInv1$\le$N$\wedge$\iInv2$\le$N$\wedge$\iInv1$\neq$\iInv2$\wedge$\\
~~f=inv$\_$$\_$3  \iInv1 \iInv2)$\vee$\\
~~($\exists$ \iInv2. \iInv2$\le$N$\wedge$f=inv$\_$$\_$4  \iInv2)$\vee$\\
~~($\exists$ \iInv1 \iInv2. \iInv1$\le$N$\wedge$\iInv2$\le$N$\wedge$\\
~~\iInv1$\neq$\iInv2$\wedge$f=inv$\_$$\_$5  \iInv1 \iInv2)\\
apply (cut$\_$tac a1, simp) done\\
moreover \{assume b1:($\exists$ \iInv1 \iInv2. \iInv1$\le$N$\wedge$\\
~~~~\iInv2$\le$N$\wedge$\iInv1$\neq$\iInv2$\wedge$f=inv$\_$$\_$1  \iInv1 \iInv2)\\
~~have formEval f s\\
~~apply (rule iniImply\_inv\_\_1)\\
~~apply (cut\_tac b1, assumption)\\
~~apply (cut\_tac a2 a3, blast) done\}\\
moreover \{assume b1:($\exists$ \iInv2. \iInv2$\le$N$\wedge$ \\
~~~~f=inv\_\_2  \iInv2)\\
~~have formEval f s\\
~~apply (rule iniImply\_inv\_\_2)\\
~~apply (cut\_tac b1, assumption)\\
~~apply (cut\_tac a2 a3, blast) done\}\\
moreover \{assume b1:($\exists$ \iInv1 \iInv2. \iInv1$\le$N$\wedge$\\
~~~~\iInv2$\le$N$\wedge$\iInv1$\neq$\iInv2$\wedge$\\
~~~~f=inv\_\_3  \iInv1 \iInv2)\\
~~have formEval f s\\
~~apply (rule iniImply\_inv\_\_3)\\
~~apply (cut\_tac b1, assumption)\\
~~apply (cut\_tac a2 a3, blast) done\}\\
moreover \{assume b1:($\exists$ \iInv2. \iInv2$\le$N$\wedge$\\
~~~~f=inv\_\_4  \iInv2)\\
~~have formEval f s\\
~~apply (rule iniImply\_inv\_\_4)\\
~~apply (cut\_tac b1, assumption)\\
~~apply (cut\_tac a2 a3, blast) done\}\\
moreover \{assume b1:($\exists$ \iInv1 \iInv2. \iInv1$\le$N$\wedge$ \\
~~~~\iInv2$\le$N$\wedge$\iInv1$\neq$\iInv2$\wedge$\\
~~~~f=inv\_\_5  \iInv1 \iInv2)\\
~~haveformEval f s\\
~~apply (rule iniImply\_inv\_\_5)\\
~~apply (cut\_tac b1, assumption)\\
~~apply (cut\_tac a2 a3, blast) done\}\\
ultimately show formEval f s by auto\\
qed\\
\end{specification}

The proof structure of {\sf lemma$\_$inv1$\_$on$\_$rules} and
{\sf invs$\_$on$\_$rules} and {\sf on$\_$inis} are also typical case analysis ones
using {\sf moreover} blocks and {\sf ultimately} commands.
Therefore, a generic program of generating a typical case analysis proof will be adopted in our framework.
\JP{How?}

%------------------------------------------------------
\medskip\noindent
{\bf The main theorem.}
%------------------------------------------------------
With the preparation of lemma on\_inis and invs\_on\_rules,
\JP{I don't know what exactly the first sentence means.}
the generation of the main lemma is quite easy.
Recall that the consistency lemma is our main weapon to prove the main lemma,
which requires proving two parts of obligations.
\JP{`weapon'? `obligations'?}

\begin{enumerate}
\item For any invariant $inv \in (\mathsf{invariants} ~N) $,  any
state $s$, if $ini$ is evaluated true at state $s$, then $inv$ is
evaluated true at state $s$. This can be solved done by applying lemma on$\_$inis.
\item For any invariant $inv \in (\mathsf{invariants} ~N)$, any $r$ in rule set
$ \mathsf{rules} ~N$ , one of the causal relations
$\mathsf{CR}_{1-3}$ holds. This can be solved done by  applying lemma invs$\_$on$\_$rules.
\end{enumerate}

\begin{specification}
\\
lemma main:\\
assumes  a1: 0<N and \\
~~a2: s$\in$ reachableSet \{andList (allInitSpecs N)\}\\
~~~~(rules N)\\
shows $\forall$ inv. inv $\in$ (invariants N) $\longrightarrow$\\
~~formEval inv s\\
proof (rule consistentLemma)\\
show consistent (invariants N)\\
~~~~\{andList (allInitSpecs N)\}\\
~~~~(rules N)\\
proof(cut\_tac a1, unfold consistent\_def,rule conjI)\\
show  $\forall$inv ini s. inv $\in$ (invariants N) $\longrightarrow$\\
~~~~ini $\in$\{andList (allInitSpecs N)\} $\longrightarrow$\\
~~~~formEval ini s $\longrightarrow$ \\
~~~~formEval inv s\\
proof((rule allI)+,(rule impI)+)\\
~~fix inv ini s\\
~~assume b1:inv $\in$ (invariants N) and\\
~~~~b2:formEval ini s and \\
~~~~b3:ini $\in$ \{andList (allInitSpecs N)\}  \\
~~show "formEval f s"\\
~~apply (rule on\_inis, cut\_tac b1, assumption,\\
~~~~cut\_tac b2, assumption, cut\_tac b3, \\
~~~~assumption) done\\
qed\\ \\

next   show  $\forall$inv r. inv $\in$ invariants N$\longrightarrow$ \\
~~~~r $\in$rules N $\longrightarrow$ \\
~~~~invHoldForRule inv r (invariants N) \\
proof((rule allI)+,(rule impI)+)\\
~~fix f r \\
~~assume b1: f $\in$ invariants N  and \\
~~~~b2:r $\in$ rules N\\
~~show invHoldForRule' s f r (invariants N)\\
~~apply (rule invs\_on\_rules, cut\_tac b1,\\
~~~~assumption, cut\_tac b2, assumption) done\\
qed\\ \\

next show s $\in$ reachableSet \\
~~~~andList (allInitSpecs N) (rules N)\\
~~apply (metis a1) done\\
qed\\
\end{specification}

The generation of the main lemma is quite easy because it is in a standard form.
\JP{Really? And for the rest of this section, I just clean as much as I can without understanding the algorytms.}

%------------------------------------------------------
\medskip\noindent
{\bf Algorithms of The Proof Generator {\sf proofGen}.}
%------------------------------------------------------
In the following, we illustrate the key techniques and algorithms of generation of
the lemmas and their proofs.
According to the order in which we present the above lemmas,
we introduce their proof generation in a bottom-up manner.
First let us introduce the generation of a subproof according to a relation tag of
${\sf CR}_{1-3}$, which is shown in Algorithm~\ref{alg:proofGenOfReltag}.

\begin{algorithm}[!t]
\caption{Generating a kind of proof which is according with a relation tag of ${\sf CR}_{1-3}$: rel2proof}
\label{alg:proofGenOfReltag}
\KwIn{A causal relation item $relTag$}
\KwOut{An Isablle proof: $proof$   }
{
 \If{${\it relTag}={\sf CR}_1$}
  {${\it proof} \leftarrow $ sprintf\\
\twoSpaces"have invHoldForRule1 f r (invariants N)  \\
\twoSpaces         by(cut\_tac a1 a2 b1, simp, auto) \\
\twoSpaces then have invHoldForRule f r (invariants N)  by blast" \; }
 \ElseIf{ ${\it relTag}={\sf CR}_2$}
  {${\it proof} \leftarrow$  sprintf\\
\twoSpaces"have invHoldForRule2 f r (invariants N)
\twoSpaces         by(cut\_tac a1 a2 b1, simp, auto) \\
\twoSpaces then have invHoldForRule f r (invariants N)  by blast" \; }
 \Else{
 \label{label:getFormField}$f' \leftarrow {\sf getFormField}({\it relTag})$\;
 ${\it proof} \leftarrow$ sprintf\\
\twoSpaces"have invHoldForRule3 f r (invariants N)  \\
\twoSpaces proof(cut\_tac a1 a2 b1, simp, rule\_tac x=\%s  in exI,auto)qed\\
\twoSpaces then have invHoldForRule f r (invariants N)  by blast" (symbf2Isabelle f')"\;}
\Return{proof}
}
\end{algorithm}

In the body of function {\sf rel2proof},
$sprintf$ writes a formatted data to string and returns it.
In line \ref{label:getFormField}, $getFormField(relTag)$ returns $f'$ if $relTag=invHoldForRule_3(f')$.
{\sf rel2proof} transforms a a relation tag into a paragraph of proof.
If the tag is among ${\sf CR}_{1-2}$,
the transformation is rather straight-forward,
otherwise the form $f'$ is assigned by the formula $getFormField(relTag)$,
and provided to tell Isabelle the formula which should be used to construct the ${\sf CR}_3$ relation.

\begin{algorithm}[!t]
\caption{Generating one sub-proof for a subcase: oneMoreOverGen}\label{alg:MoreOver}

\KwIn{A formula $caseFsm$ standing for the assumption of the subcase,
a relation item $relItem$ containing the information of causal relation }
\KwOut{  An Isablle proof: $subProof$ }
{
${\it proof} \leftarrow {\sf rel2proof}(relItem)$\;
${\it subProof} \leftarrow$  sprintf\\
\twoSpaces "moreover\{assume  b1:\%s  \\
           \%s    \} "\\
\twoSpaces ( asm, proof)\;
\Return{subproof}
}
\end{algorithm}

In Algorithm~\ref{alg:MoreOver}, {\sf oneMoreOverGen} generates a subproof
for a subcase in a proof of case analysis.
It returns a subproof which is composed by filling an assumption of the subcase such as
"\iR1=\iInv1" and a paragraph of proof generated by ${\sf rel2proof}(relItem)$
into a format of block {\tt morover \{ ...\}}.

Due to the common use of case analysis proof of using {\sf moreover} and {\sf ultimately} commands,
we design a generic program for performing case analysis {\sf doCaseAnalz}.
In algorithm~\ref{alg:doCaseAnalz}, formulas standing for case-splitting $partition$,
subproofs $subproofs$, and the conclusion $concluding$ are needed  in case analysis to fill the format.

\begin{algorithm}[!t]

\caption{Generating a whole proof of doing case analysis: doCaseAnalz}\label{alg:doCaseAnalz}
\KwIn{A formula $partition$ standing for case-splittings, a proof list $subproofs$ standing  all the subproofs of each subcases, concluding parts $concluding$}
\KwOut{  An Isabelle proof: $proof$ }
{
${\it proof} \leftarrow $sprintf\\
\twoSpaces \twoSpaces  " have \%s  by auto\\
\twoSpaces \twoSpaces         \%s\\
\twoSpaces \twoSpaces        ultimately show \%s by auto"\\
\twoSpaces     (partition, subproofs,  concluding) \;
\Return{proof}
}
\end{algorithm}


In Algorithm~\ref{alg:doCaseAnalzI},
{\sf caseAnalzI} generates a typical proof of doing case analysis to
prove some causal relation hold between some rule and invariant.
${\sf oneMoreOverGenI}({\it case},{\it rel})$ formula comes from the disjunction of formulas
in the {\tt symbCases} field of $rec$,
which is returned by $caseField(rec)$,
subproofs $subproofs$ are generated by concatenation of all the subproofs,
each of which is generated by ${\sf oneMoreOverGenI}({\it case},{\it rel})$.
The proof is simply composed by calling ${\sf doCaseAnalz}({\it partition},{\it subproofs},{\it concluding})$.

\begin{algorithm}[!t]
\caption{Generating a whole proof of doing case analysis on parameters of rule and invariant: caseAnalzI}\label{alg:doCaseAnalzI}
\KwIn{A record $rec$ fetched from $symbCausal$ }
\KwOut{  An Isablle proof: $proof$ }
{
${\it cases} \leftarrow {\sf caseField}(rec)$\;
${\it rels} \leftarrow {\sf relItems}(rec)$;
${\it partition} \leftarrow \bigvee {\it cases}$\;
${\it subproofs} \leftarrow ""$\;
\While{$({\it cases} \ne [])$}
{ $ {\it case} \leftarrow {\sf head}({\it cases})$ \;
  ${\it cases} \leftarrow {\sf tail}({\it cases})$ \;
  $ {\it rel} \leftarrow {\sf head}({\it rels})$ \;
  ${\it rels} \leftarrow {\sf tail}({\it rels})$ \;
  ${\it subproofs} \leftarrow {\it subproofs} \cat {\sf oneMoreOverGenI}({\it case},{\it rel})$\;
  }
${\it concluding} \leftarrow $"invHoldForRule s f r (invariants N) "\;
${\it proof} \leftarrow {\sf doCaseAnalz}({\it partition},{\it subproofs},{\sf concluding})$\;
\Return ${\it proof}$
}
\end{algorithm}

Next we discuss how to generate assumptions on an invariant formula of an lemma such as $critVsInv1$.
In the body of Algorithm~\ref{alg:asmGenOnInv},
${\sf tbl\_element}({\it symbInvs}, {\it invName})$  retrieves the record on a invariant formula from
${\it symbInvs}$ to ${\it invItem}$
by its name ${\it invName}$, ${\sf invParaNum}({\it invItem})$
and ${\sf constrOfInv}({\it invItem}))$ return the field ${\tt invNumFld}$ and ${\tt constr}$ of  ${\it invItem}$, respectively.
${\sf invParasGen}({\it lenPInv})$ generates a string of a list of actual parameters such as
${\it iInv_1} \ldots {\it iInv_{lenPInv}}$ if ${\it lenPInv}>0$,
else an empty string $""$.
At last, the assumption on the invariant is created by filling ${\it invParas}$,
${\it constrOnInv}$, and ${\it invName}$ into a proper place in the format if needed.

\begin{algorithm}[!t]
\caption{Generating an assumption on an invariant formula: asmGenOnInv}\label{alg:asmGenOnInv}
\KwIn{An invariant name ${\it invName}$,    a table ${\it symInvs}$ storing invariant formulas   }
\KwOut{  An assumption on an invariant formula: $asm$   }

${\it invItem} \leftarrow {\sf tbl\_element}( {\it symbInvs}, {\it invName})$\;
${\it lenPInv} \leftarrow {\sf invParaNum}({\it invItem})$\;
${\it invParas} \leftarrow {\sf invParasGen}({\it lenPInv})$\;
${\it constrOnInv} \leftarrow {\sf symbForm2Isabelle}({\sf constrOfInv}({\it invItem}))  $\;
 \If {{\it lenPInv} = 0}
  {$asm  \leftarrow  "a1: f="\cat invName   $\;}
  \Else{$asm  \leftarrow$ sprintf "a1: $\exists$ \%s. \%s $\wedge$ f=\%s \%s" (invParas,  constrOnInv, invName, invParas)\;}
  \Return{asm}
\end{algorithm}

Similar to {\sf asmGenOnInv}, {\sf obtainGenOnInv},
which is shown in Algorithm~\ref{alg:obtainGenOnInv},
generates a proof command of {\sf obtain} by retrieving and
generating the related information and filling them in a format on {\sf obtain}.
Similar to {\sf asmGenOnInv} and  {\sf obtainGenOnInv},
{\sf asmGenOnRule} and  {\sf obtainGenOnRule} generate an  assumption
and {\sf obtain} proof command   on a rule.

\begin{algorithm}[!t]
\caption{Generating an {\sf obtain} proof command on an invariant formula: obtainGenOnInv}\label{alg:obtainGenOnInv}
\KwIn{An invariant name ${\it invName}$,    a table ${\it symInvs}$ storing rules    }
${\it invItem}   \leftarrow {\sf tbl\_element}({\it symbInvs,  invName})$\;
${\it lenPInv} \leftarrow {\sf invParaNum}({\it invItem})$\;
${\it invParas} \leftarrow {\sf invParasGen}({\it lenPInv})$\;
\If {$  {\it lenPInv} = 0 $}
    {${\it obtain} \leftarrow  ""$\;}
\Else {${\it obtain} \leftarrow$ sprintf "from a1 obtain  \%s where a1:\%s $\wedge$ f=\%s \%s by auto"\\
 \twoSpaces  ${\it (invParas,  constrOnInv, invName, invParas)}$\;}
 \Return{{\it obtain}}
\end{algorithm}

After the above functions, now the generation of a lemma on the causal relation
is rather easy, which is shown in Algorithm~\ref{alg:lemmaOnCausalRuleInv}.
After generating an assumption on invariant formula ${\it asm1}$,  ${\it asm2}$ on a rule,
an obtain command  ${\it obtain1}$ on the invariant,
and ${\it obtain2}$ on the rule,  ${\it symRelItem}$ is retrieved from ${\it symCausalTab}$
by ${\it ruleName}\cat {\it invName}$,
and a proof ${\it proof}$ is generated by calling ${\sf caseAnalzI}({\it symRelItem})$.
At last these parts are filled into proper places in the lemma format.

\begin{algorithm}[!t]
\caption{Generating a lemma on a causal relation: lemmaOnCausalRuleInv}
\label{alg:lemmaOnCausalRuleInv}
\KwIn{A parameterized rule name ${\it ruleName}$,
a formula name ${\it invName}$,
a table ${\it symRules}$ storing rules,
a table ${\it symInvs}$ storing invariant formulas,
a table ${\it symCausalTab}$  storing causal relation  }
\KwOut{  An Isablle proof script for a lemma: ${\it lemmaWithProof}$   }
{
 ${\it asm1} \leftarrow {\sf asmGenOnInv}({\it symbInvs},{\it invName})$\;
 ${\it asm2} \leftarrow {\sf asmGenOnRule}({\it symbRules},{\it ruleName})$\;
 ${\it obtain1} \leftarrow {\sf obtainGenOnInv}({\it symbInvs},{\it invName})$\;
 ${\it obtain2}\leftarrow {\sf obtainGenOnRule}({\it symbRules},{\it ruleName})  $\;
 ${\it symRelItem} \leftarrow {\sf tbl\_element}({\it symCausalTab},({\it ruleName}\cat {\it invName}))$\;
 ${\it proof} \leftarrow {\sf caseAnalzI}({\it symRelItem})$\;
 ${\it lemmaWithProof} \leftarrow$ sprintf \\
\twoSpaces"lemma \%sVs\%s:\\
\twoSpaces assumes \%s and \%s\\
\twoSpaces  shows  invHoldForRule s f r (invariants   N)\\
\twoSpaces  proof -
\twoSpaces \%s~ \%s~  \%s
\twoSpaces qed"\\
\twoSpaces ${\it (ruleName, invName, asm1,asm2,}$ \\
\twoSpaces\twoSpaces ${\it obtain1, obtain2, proof)}$ \;
    \Return ${\it lemmaWithProof}$
}
\end{algorithm}

Due to page limitation,
we illustrate the algorithm  for generating  a key part of the proof of the lemma {\tt critVsinv1}:
the generation of a subproof (e.g., lines 7-8) according to a symbolic  relation tag of $\mathsf{CR}_{1-3}$,
which is shown in Algorithm~\ref{alg:proofGenOfReltag}.
Input ${\it relTag}$ is the result of the generalization step,
which is discussed in Section~\ref{sec:generalization}.

\begin{algorithm}[!t]
\caption{Generating a kind of proof which is according with a relation tag of ${\sf CR}_{1-3}$: rel2proof}
\label{alg:proofGenOfReltag}
\KwIn{A symbolic causal    relation item ${\it relTag}$}
\KwOut{  An Isablle proof: ${\it proof}$   }
{
 \If{${\it relTag}={\sf CR}_1$}
  {${\it proof} \leftarrow $ sprintf\\
\twoSpaces"have invHoldRule1 f r (invariants N)  \\
\twoSpaces         by(cut\_tac a1 a2 b1, simp, auto) \\
\twoSpaces then have invHoldRule f r (invariants N)  by blast" \; }
 \ElseIf{ ${\it relTag}={\sf CR}_2$}
  {${\it proof} \leftarrow$  sprintf\\
\twoSpaces"have invHoldRule2 f r (invariants N)
\twoSpaces         by(cut\_tac a1 a2 b1, simp, auto) \\
\twoSpaces then have invHoldRule f r (invariants N)  by blast" \; }
 \Else{
 \label{label:getFormField}$f' \leftarrow {\sf getFormField}({\it relTag})$\;
 ${\it proof} \leftarrow$ sprintf\\
\twoSpaces"have invHoldRule3 f r (invariants N)  \\
\twoSpaces proof(cut\_tac a1 a2 b1, simp, rule\_tac x=\%s  in exI,auto)qed\\
\twoSpaces then have invHoldRule f r (invariants N)  by blast" (symbf2Isabelle f')"\;}
\Return{proof}
}
\end{algorithm}

In the body of function {\sf rel2proof},  $\mathsf{sprintf}$ writes a formatted data to string and returns it.
In Line~\ref{label:getFormField}, $\mathsf{getFormField}(relTag)$ returns the field of formula $f'$,
if $relTag=\mathsf{invHoldRule_3}(f')$.
{\sf rel2proof} transforms a symbolic relation tag into a paragraph of proof, as shown in lines 7-8, 10-11, or 13-14.
If the tag is among $\mathsf{invHoldRule_{1-2}}$,
the transformation is rather straight-forward, else the form $f'$ is assigned
by the formula $\mathsf{getFormField(relTag)}$, and provided to tell Isabelle the formula
which is used to construct the $\mathsf{CR}_3$ relation.

\JP{In the end, I feel that it might be better to explain everything `top-down'.
In many of the presented algorithms, there are many functions are not explained well.
Maybe the whole idea of proof generation is simple (as syntax transformation),
but this whole section is quite messy, in a way that it mixes information texts and formalisations that we try to
achieve in previous sections.}

